\documentclass[aps,pra,notitlepage,amsmath,amssymb,letterpaper,12pt]{revtex4-1}
\usepackage{amsthm}
\usepackage{graphicx}
 
%  Helpful commands to set up problem environments easily
\newenvironment{problem}[2][Problem]{\begin{trivlist}
\item[\hskip \labelsep {\bfseries #1}\hskip \labelsep {\bfseries #2.}]}{\end{trivlist}}
\newenvironment{solution}{\begin{proof}[Solution]}{\end{proof}}
 
% --------------------------------------------------------------
%                   Document Begins Here
% --------------------------------------------------------------
 
\begin{document}
 
\title{Homework X}
\author{Jo Student}
\affiliation{PHYS 220 -- Scientific Computing I}
\date{\today}

\maketitle

% x.yz is the problem number
\begin{problem}{x.yz} 
Problem statement goes here.
\end{problem}
 
\begin{solution} %You can also use solution in place of proof.
Treat equations as part of complete sentences.  For example, the Schr\"odinger equation can be written in the position representation as
% Use align environments for equations.  align* removes equation numbers
\begin{align}
i\hbar \partial_t \psi(x,t) & = \hat{H}\psi(x,t) \\
& = -\frac{\hbar^2}{2m}\nabla^2\psi(x,t) + V(x)\psi(x,t). \nonumber % removes numbers by line
\end{align}

Figures can be included easily.

\begin{figure}[h!] % h forces the figure to be placed here, in the text
  \includegraphics[width=0.4\textwidth]{stormtroopocat.jpg}  % if pdflatex is used, jpg, pdf, and png are permitted
  \caption{The figure caption goes here.}
  \label{fig:figlabel}
\end{figure}

This text should be below the figure unless \LaTeX decides that a different layout works better.
\end{solution}
 
% Repeat as needed
 
 
\end{document}